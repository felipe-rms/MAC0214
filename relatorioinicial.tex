\documentclass{article}
\usepackage{graphicx} % Required for inserting images
\usepackage{amssymb}
\usepackage{hyperref}
\usepackage{amsthm}
\usepackage{setspace}
\usepackage[left=3cm,top=2cm,bottom=2cm,right=3cm]{geometry}
\usepackage{authblk}
\usepackage[brazilian]{babel}

\title{Projeto de MAC0214}
\author[1]{
    Aluno: Felipe Ramos
    \texttt{feliperamos@usp.br} \\
    Orientador: Guillermo Junchaya 
    \texttt{enriquejh@usp.br}
}

\date{\today}

\begin{document}

\maketitle

\vspace{6cm}

\tableofcontents  % Cria o índice

\pagebreak

\section{Resumo do projeto}
\subsection{Introdução}
\hspace{1cm}A programação competitiva é uma competição que envolve a resolução de problemas lógicos e matemáticos. Os competidores devem usar suas habilidades em programação e seus conhecimentos de disciplinas de computação.

\hspace{5mm}No IME-USP, o MaratonUSP se destaca com equipes vencedoras na maratona de programação da Sociedade Brasileira de Computação (SBC). Esse projeto, inspirado na disciplina desafios de programação oferecida por competidores do maratonusp, visa o preparo para as provas da SBC.
\subsection{Objetivos}
\hspace{1cm} O projeto tem como fim a preparação para a fase 2 da maratona da SBC. Espera-se que os participantes adquiram experiência nos assuntos abordados, bem como em ambientes de programação competitiva.

\section{Metodologia}
\hspace{1cm}A cada semana será abordado um assunto diferente, e o aprendizado se dará por estudo da teoria e resolução de problemas.

\hspace{0.5cm}Além dos assuntos semanais, todos os sábados os participantes se reunirão para resolver contests online. Nessa etapa poderá ser observada a evlução dos competidores em provas sem classificação de tema.

\hspace{5mm} As soluções de todos os exercícios feitos será disponibilizada em \href{https://github.com/felipe-rms}{https://github.com/felipe-rms}
\section{Cronograma}
\hspace{1cm}Ao longo de 12 semanas serão estudados os seguintes assuntos, em um total de 4 horas para cada:
\begin{enumerate}
    \item 2-Pointers
    \item BFS e Djikstra
    \item Programação Dinâmica
    \item SegTree
    \item Teoria dos Números
    \item Line sweep
    \item Geometria
    \item Trie
    \item Suffix-Array
    \item Matchings Máximos
    \item Fluxos
    \item Combinatória
\end{enumerate}

\hspace{5mm}Aos sábados serão realizados contests na plataforma \href{https://codeforces.com}{codeforces}, totalizando mais 4 horas semanais. Por fim, devemos ainda participar de pelo menos duas provas: a seletiva USP e a fase 1 da maratona SBC, cada uma com 7 horas de duração. No final do projeto, serão cumpridas 110 horas de estudo, resolução de problemas e participação em provas. 

\section{Relatório final}
\subsection{Atividades}
\hspace{1cm} O projeto foi realizado ao longo do semestre como planejado, e os estudos e competições foram registrados no repositório público MAC0214 no GitHub. As fontes de problemas utilizadas foram a planilha de maratona da UFMG, o site CSES, que é um banco de questões divididas por tema, o Codeforces e o Atcoder, sites que realizam as competições online e liberam editoriais para os problemas. Ademais, as fontes de estudo teórico foram vídeos sobre os temas no YouTube, slides de aulas do grupo de maratona da UFMG, e sites educativos sobre proramação como o GeeksforGeeks e o Stack Overflow. As competições online foram todas individuais, enquanto as competições presenciais foram realizadas em trio.

\subsection{Resultados}
\hspace{1cm} Houve uma notável evolução do aluno ao longo da realização das atividades, onde foi possível melhorar em aspectos como a escrita de código, o conhecimento de estruturas de dados, o domínio de algoritmos, habilidades matemáticas e o uso de ferramentas da linguagem C++. Além disso, houve uma melhora nas habilidades comunicativas, pela realização de competições em trio, onde o aluno pôde interagir com os colegas de equipe, aumentando suas capacidades de comunicar as ideias de problemas e desenvolver soluções em grupo.

\subsection{Comentários}
\hspace{1cm}  Durante o desenvolvimento do projeto, houve algumas mudanças nas escolhas dos temas estudados, considerando a relevância e a interdisciplinaridade com outras disciplinas cursadas no mesmo semestre. Da mesma forma, alguns temas foram estendidos para duas semanas de duração, devido à quantidade de horas dedicadas aos estudos teóricos e ao número de exercícios resolvidos. Em média, foram solucionadas cerca de 6 questões por tema a cada semana.

\hspace{1cm}Ademais, o aluno participou de 11 competições (contests) ao longo do semestre, onde foram registrados o desempenho em cada um deles e as horas dedicadas, dado o tempo de aquecimento, realização da competição, e solução de questões após o fim do contest (upsolving).


\subsection{Relatório de Horas Cumpridas}
A carga horária dedicada ao projeto foi cuidadosamente monitorada. Existe um detalhamento mais específico referente às horas gastas em cada semana no README do repositório desse projeto no GitHub. De todo modo, abaixo está um panorama do tempo investido em diferentes atividades:

\begin{itemize}
    \item \textbf{Estudos teóricos e resolução de exercícios por terma} 72 horas.
    \item \textbf{Participação em competições online:} 25 horas.
    \item \textbf{Participações em competições presenciais:} 15 horas.
    \item \textbf{Carga horária total aproximada:} 112 horas.
\end{itemize}

A carga horária total cumprida ao longo do semestre foi de aproximadamente 112 horas, refletindo o comprometimento e a consistência no desenvolvimento do projeto.


\end{document}
